\section*{Appendix: Detailed Implementation Spec\footnote{Get source code in \url{https://github.com/ktaebum/NC-GAN}}}
\label{sec:appendix}

\subsection*{Models Package}

\begin{itemize}
	\item patch\_gan.py
	\begin{itemize}
		\item PatchGAN discriminator class 포함
	\end{itemize}
	\item pix2pix.py
	\begin{itemize}
		\item 기존 \pixpix~U-Net generator class 포함
	\end{itemize}
	\item style2paint.py
	\begin{itemize}
		\item \stylepaint 에 기반한 generator class 포함
	\end{itemize}
\end{itemize}

\subsection*{Preprocess Package}
\begin{itemize}
	\item image.py
	\begin{itemize}
		\item def save\_image(image, filename, path='.')
		\begin{itemize}
			\item PIL 이미지 객체와 filename string, path string을 argument로 받는다
			\item 지정된 path에 이미지를 png 포맷으로 저장한다
		\end{itemize}
		\item def centor\_crop\_tensor(image, size=224)
		\begin{itemize}
			\item PIL 이미지 객체와 사이즈 (정수)를 argument로 받는다
			\item 이미지의 중심을 기준으로 자른 이미지를 반환한다
		\end{itemize}
		\item def scale(image)
		\begin{itemize}
			\item torch Tensor 이미지 객체를 argument로 받는다
			\item 이미지는 픽셀 값이 $[0, 1]$에 바운드 되어있다.
			\item 이미지의 각 픽셀 값이 $[-1, 1]$에 바운드 되도록 스케일한다
		\end{itemize}
		\item def re\_scale(image)
		\begin{itemize}
			\item torch Tensor 이미지 객체를 argument로 받는다
			\item 이미지는 픽셀 값이 $[-1, 1]$에 바운드 되어있다.
			\item 이미지의 각 픽셀 값이 $[0, 1]$에 바운드 되도록 스케일한다
		\end{itemize}
		\item def grayscale\_tensor(images, device)
		\begin{itemize}
			\item torch Tensor 이미지 객체와 torch device (cuda or cpu)를 argument로 받는다
			\item 칼라 이미지를 흑백 이미지로 바꾸고 흑백 이미지를 반환한다
		\end{itemize}
	\end{itemize}
	\item niko.py
	\begin{itemize}
		\item 학습 데이터셋을 불러오는 NikoPairedDataset 클래스 포함
	\end{itemize}
	\item sketch.py
	\begin{itemize}
		\item def get\_sketch(image, smooth='basic', smooth\_iter=1)
		\begin{itemize}
			\item 채색된 PIL 이미지 객체, smoothing 옵션 (no, basic, more), smoothing 횟수를 argument로 받는다
			\item PIL edge detection을 통해 얻어낸 스케치 이미지를 반환한다
		\end{itemize}
	\end{itemize}
\end{itemize}


\subsection*{Trainer Package}
\begin{itemize}
	\item trainer.py
	\begin{itemize}
		\item 모델 학습을 하는 trainer의 abstraction class 포함
	\end{itemize}
\end{itemize}


\subsection*{Utils Package}

\begin{itemize}
	\item args.py
	\begin{itemize}
		\item Command line argument parser를 반환한다
	\end{itemize}
	\item average.py
	\begin{itemize}
		\item 학습시, loss 값의 평균치를 추적하는 AverageTracker class 포함
	\end{itemize}
	\item image.py
	\begin{itemize}
		\item 안정된 학습에 도움이 된다고 알려진 ImagePooling class 포함
	\end{itemize}
	\item io.py
	\begin{itemize}
		\item def save\_checkpoints(model,
		save\_name=None,
		epoch=None,
		evaluation=None,
		optimizer=None)
		\begin{itemize}
			\item 학습된 모델과 optimizer를 저장한다
		\end{itemize}
		\item def load\_checkpoints(checkpoint, model, optimizer=None)
		\begin{itemize}
			\item 저장된 기존 학습모델과 optimizer를 불러온다
		\end{itemize}
	\end{itemize}
	\item losses.py
	\begin{itemize}
		\item GANLoss를 계산하는 class 포함
	\end{itemize}
\end{itemize}