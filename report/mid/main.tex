\documentclass[runningheads]{llncs}

%% Save the class definition of \subparagraph
\let\llncssubparagraph\subparagraph
%% Provide a definition to \subparagraph to keep titlesec happy
\let\subparagraph\paragraph
%% Load titlesec
\usepackage{titlesec}
%\titleformat{\section}{\normalfont\bfseries}{\thesection}{1em}{}
\titlespacing*{\section}{0pt}{*1.5}{*1.5}
\titlespacing*{\subsection}{0pt}{*1.3}{*1.3}
\titlespacing*{\subsubsection}{0pt}{*1.0}{*1.0}
%% Revert \subparagraph to the llncs definition
\let\subparagraph\llncssubparagraph

\usepackage{graphicx}
\usepackage{tikz}
\usepackage{float}
\usepackage{booktabs}
\usepackage{amsmath}
\usepackage{amssymb}
\usepackage{blindtext}
\usepackage{tabularx}
\usepackage[utf8]{inputenc}
\usepackage{url}

\usepackage{wasysym}
\usepackage{multirow}

%\usepackage{multicol}

\usepackage{algorithm}
\usepackage{algorithmic}
\renewcommand{\algorithmicrequire}{\textbf{Input:}}
\renewcommand{\algorithmicensure}{\textbf{Output:}}

\newcolumntype{L}[1]{>{\raggedright\arraybackslash}p{#1}}
\newcolumntype{C}[1]{>{\centering\arraybackslash}p{#1}}
\newcolumntype{R}[1]{>{\raggedleft\arraybackslash}p{#1}}

\def\mathbi#1{\textbf{\em #1}}
\graphicspath{{./images/}}

\usetikzlibrary{shapes, arrows}

%tikz style
\tikzstyle{decision} = [diamond, draw,
    text width=7em, text badly centered, node distance=3cm, inner sep=0pt]
\tikzstyle{block} = [rectangle, draw,
    text width=6em, text centered, rounded corners, minimum height=4em]
\tikzstyle{line} = [draw, -latex']
\tikzstyle{cloud} = [draw, ellipse, node distance=3cm,
    minimum height=2em]
\tikzstyle{arrow} = [thick,->,>=stealth]

\newcommand{\red}[1]{\textcolor{red}{#1}}
\newcommand{\blue}[1]{\textcolor{blue}{#1}}
\newcommand{\methodlong}[0]{Adversarial Layerwise Compression}
\newcommand{\method}[0]{\textsc{AlCom}\xspace}
\DeclareMathOperator*{\argmin}{arg\,min}

%\titlespacing*{\section}{0pt}{1.1\baselineskip}{\baselineskip}

%\setlength{\floatsep}{0.05cm}
%\setlength{\textfloatsep}{0.05cm}
%\setlength{\intextsep}{0.05cm}
%\setlength{\dblfloatsep}{0.05cm}
%\setlength{\dbltextfloatsep}{0.05cm}
%\setlength{\abovecaptionskip}{0.05cm}
%\setlength{\belowcaptionskip}{0.05cm}
%\setlength{\abovedisplayskip}{0.05cm}
%\setlength{\belowdisplayskip}{0.05cm}




\begin{document}
\title{Adversarial Layerwise Compression of Deep Neural Networks
}
%\titlerunning{Abbreviated paper title}
% If the paper title is too long for the running head, you can set
% an abbreviated paper title here

%\author{Taebum Kim}
%
%\authorrunning{Kim}
%\institute{Seoul National University \\
%\email{k.taebum@snu.ac.kr}}

\maketitle              % typeset the header of the contribution

\begin{abstract}

Given a neural network, how can we find low-rank approximations of its hidden layers with tolerable loss of accuracy?
Compressing neural networks is a crucial task for deploying recent models in mobile platforms or embedded systems that have limited resources.
A tensor decomposition is one of the most successful ways of compressing neural networks and has been recently exploited for compressing various types of convolutional or recurrent neural networks.
In this paper, we propose \methodlong{} (\method{}), a novel approach to compress a neural network by the tensor decomposition and layerwise knowledge distillation.
\method{} is a general algorithm that is applicable to any neural network with any tensor decomposition model.
Our extensive experiments show that \method{} consistently outperforms the state-of-the-art method based on Tucker decomposition on various models and datasets for classifying images such as MNIST and CIFAR100.

\keywords{Neural network \and Model compression \and Tensor decomposition \and Knowledge distillation \and Adversarial training}
\end{abstract}


\section{Introduction}
test


\section{Related Works}



\input{030method}

\input{040experiment}

\input{050conclusion}

\bibliographystyle{splncs04}
\bibliography{bib}
\end{document}
