\section{Goal/Problem \& Requirements}

본 프로젝트에서는 현재까지 알려진 채색 모델 중 가장 좋은 성능을 보이는 Style2Paint 모델을 기반으로 하여 AutoPainter 모델을 능가하는 것을 목표로 한다.
Style2Paint 모델은 높은 정확도의 채색 성능을 보이고 있으나, 기본적인 아이디어만 제시되어 있을 뿐 세부적인 구현 내용은 공개되어 있지 않다.
현실적으로 Sytle2Paint의 결과를 100\% 구현하기에는 어려움이 있으므로 다른 초기의 자동 채색 모델 중 비교적 좋은 성능을 보여주는 AutoPainter 모델을 비교 대상으로 설정하였다.

Style2Paint 모델은 U-Net 구조의 generator, VGGNet 기반 Style Extractor, ACGAN의 auxiliary classification discriminator를 사용한다.
이 때 학습과정에서는 스케치와 채색된 이미지의 데이터 쌍이 필요하다.
학습에 필요한 데이터 셋을 바로 구하기는 어렵기 때문에 채색된 이미지로부터 스케치를 추출해내는 과정이 필요하다.
또한 학습에 필요한 모델의 세부 구조와 하이퍼파라미터 값이 정확하게 알려져 있지 않다.
이러한 부분들은 채색 결과에 큰 영향을 미치므로, 좋은 성능을 내도록 이를 최적화하는 과정이 필요하다.
마지막으로 Style2Paint 모델은 style extraction 과정에서 classification 모델을 사용하기 때문에 원하는 스타일의 색 정보가 정확하게 반영되지 못 하는 점이 한계로 지적된다.

본 프로젝트에서는 이러한 문제들을 해결하기 위하여 다음과 같은 세부 목표들을 설정하였다.
\begin{itemize}[topsep=0pt,itemsep=-1ex,partopsep=1ex,parsep=1ex]
	\item 스케치 - 채색 학습 데이터셋의 생성
	\item Style2Paint 모델의 구현 및 최적화
	\item 효율적인 스타일 추출 모델 반영
\end{itemize}