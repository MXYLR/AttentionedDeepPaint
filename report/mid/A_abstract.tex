\section{Abstract}

최근 딥러닝 기술의 발전으로 사람이 해오던 여러 분야에서 컴퓨터가 그 역할을 대체하기 시작했다. Generative Adversarial Network (GAN) \cite{Goodfellow2014}을 이용한 자동 스케치 채색 기술은 웹툰 분야에서 놀라운 생산성의 향상으로 이어질 것으로 전망되며, 웹툰뿐만 아니라 게임, 애니메이션 산업 등에서도 널리 활용될 것으로 기대된다. 본 프로젝트에서는 현재 스케치 채색 분야에서 가장 높은 성능의 결과를 보여주고 있는 style2paint 모델 \cite{Zhang2017}을 기반으로 하여 더 나은 성능을 보이는 새로운 채색 모델을 구현하는 것을 그 목표로 한다. 현재까지는 style2paint 모델을 재 구현하는 것에 초점을 맞추어 프로젝트를 진행하였다. 
%Niko 데이터셋으로부터 배경이 단순하고 인물 한 명이 채색되어 있는 그림들을 약 1000장 가량 선별하였으며,이이PIL의 edge detection 알고리즘으로 학습을 위한 스케치와 채색 이미지쌍을 제작하였다. 
학습 데이터셋과 실험 데이터셋을 선별하고, 학습 데이터셋은 Pytorch \cite{pytorch}를 기반으로 구현된 style2paint 모델에서 학습되었다. 이후 실험 데이터셋의 실험을 통해 채색을 잘 해내는 것을 확인하였다.
하지만 현재 스케치 채색이 이루어지기는 하지만 target style를 명확하게 잘 반영하지 못하는 한계점이 있다. 추후 style을 반영된 채색이 될 수 있도록 현재 모델의 구조를 수정하여 채색 정확도를 높일 계획이다.	